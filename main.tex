%----------------------------------------------------------------------------------------
%	PACKAGES AND OTHER DOCUMENT CONFIGURATIONS
%----------------------------------------------------------------------------------------

\documentclass[10pt,a4paper,twoside]{memoir} % Change font size here (allowable values are 9pt-12pt), change the paper size, specify one or two sided printing and specify whether to show trimming lines

\input{structure.tex} % Include the file containing the code defining the structure and style of the document

%------------------------------------------------
% Thesis Information

\title{Development and Testing of an Optimal Cardiometabolic Genetic Risk Score to Predict Coronary Artery Disease Risk} % Thesis title

\author{Christopher B. Cole} % Author name

\date{May 2016} % The date

\newcommand{\institution}{University of Ottawa\xspace} % University/institution name

\newcommand{\department}{Department of Biology\xspace} % Department name

%------------------------------------------------
% Fonts

\renewcommand*{\acffont}[1]{{\normalsize\itshape #1}} % Font style for the acronym text (e.g. Do It Yourself)
\renewcommand*{\acfsfont}[1]{{\normalsize\upshape #1}} % Font style for the acronym in bracket (e.g. (DIY))

%------------------------------------------------
% Hyphenations

\hyphenation{a-no-ma-lous a-no-ma-ly amounts breaches} % Specify custom hyphenation points in words with dashes where you would like hyphenation to occur, or alternatively, don't put any dashes in a word to stop hyphenation altogether

%----------------------------------------------------------------------------------------
%	TITLE PAGE
%----------------------------------------------------------------------------------------

\renewcommand{\maketitlehooka}{
\centering
\includegraphics[width=2.5cm]{Figures/uottawa}\\[.5cm] % Institution logo
\institution\\ % Print institution name
\emph{\department}\\[.2cm] % Print department name
HONOURS B.Sc. BIOMEDICAL SCIENCE, OPTION IN BIOSTATISTICS % Degree or other information
\par
\hrulefill
\vfill}
\renewcommand{\maketitlehookb}{\vfill}
\renewcommand{\maketitlehookc}{
\vfill
\begin{flushleft}
Thesis Supervisor:\\
\textbf{Prof. Ruth McPherson, \tiny{MD, PhD, FACP, FRCPC, FRSC}}\\[.3cm] % Advisor's/supervisor's name
Secondary Thesis Supervisor:\\
\textbf{Dr. Majid Nikpay, \tiny{PhD}} % Doctoral program supervisor's name
\end{flushleft}
\vfill}
\preauthor{\begin{flushright}Honours Dissertation of:\\\bfseries} % Text prior to the author name - right aligned and bold
\postauthor{\end{flushright}} % After the author name, stop right alignment

%----------------------------------------------------------------------------------------

\makeindex % Write an index file

\begin{document}

\begin{titlingpage}
\maketitle % Print the title page
\end{titlingpage}

\frontmatter % Use roman page numbering style (i, ii, iii, iv...) for the pre-content pages

%----------------------------------------------------------------------------------------
%	PREFACE
%----------------------------------------------------------------------------------------

\section*{Preface}

Fill in later

\begin{flushright}
\textsc{\theauthor}\\
Ottawa\\
May 2016
\end{flushright}

\cleartoverso % Force a break to an even page

%----------------------------------------------------------------------------------------
%	ABSTRACT
%---------------------------------------------------------------------------------------

\begin{abstract}

\textbf{Background and Rationale}: Coronary artery disease (CAD) is a major cause of morbidity and mortality and much international effort has been expended to detect risk factors, both heritable and environmental. Although there is a well established genetic basis for CAD, genome wide association studies (GWAS) have identified just 46 common loci, explaining only a small fraction (~13\%) of the predicted heritability of CAD, estimated by twin studies to be between 40 and 60\%.  This “missing heritability”, may be explained by diverse phenomenon including multiple common variants of very low effect size that may act via multiple causal risk factors for CAD and escape detection in sample sizes investigated to date, rare variants (MAF < 1\%) of high effect size, gene × gene (G×G) interactions, and gene × environment (G $\times$ E) interactions. Previous efforts have tested the ability of a genetic risk score based on from 13 to 30 CAD-associated single nucleotide polymorphisms (SNPs) to predict CAD risk.  Even this small number of risk alleles was shown to have significant predictive power and recently, to identify individuals who would benefit most from statin therapy to reduce LDL concentrations. However, improvements in genetic risk assessment are necessary and feasible given recent genetic advancements.f
 \newline

\textbf{Purpose and Specific Objectives}: This study hopes to develop an improved genetic risk score for coronary artery disease using a panel of independent risk loci. We address whether or not a panel of 202 independent SNPs with stepwise addition of cardiometabolic condition SNPs significantly predicts CAD. \newline

\textbf{Materials and Methods}: 202 Independent SNPs were identified through GWAS and linear regression with multidimensional scaling in PLINK. The present study will use a stepwise logistic regression model with principal components and additional covariates. The independent variable will be a composite of genetic risk equal to a weighted sum of risk alleles with mean value imputation. The study will also compute Nagelkerke’s Psuedo-R2 as a proxy measure for goodness of fit of the model. Additionally, we will compute the receiver operator characteristic curve and calculate the area under the curve to determine model predictive accuracy. The net recombination index will also be calculated for each model. Accurate multiple correction will be performed with respect to the correlation matrix between tests. Additionally, the above analysis will be repeated using different FDR thresholds using the R program PRSice. \newline

\textbf{Results}: This study will result in several metrics describing the model's ability to predict CAD in a population. If the predictive ability of our score is meaningful, it will allow clinical researchers to diagnostically determine individual risk to CAD. 





\end{abstract}

\cleartoverso % Force a break to an even page

%----------------------------------------------------------------------------------------
%	TABLE OF CONTENTS
%----------------------------------------------------------------------------------------

\tableofcontents* % Print the table of contents

\cleartoverso % Force a break to an even page

%----------------------------------------------------------------------------------------
%	LIST OF FIGURES
%----------------------------------------------------------------------------------------

\listoffigures % Print the list of figures

\cleartoverso % Force a break to an even page

%----------------------------------------------------------------------------------------
%	LIST OF TABLES
%----------------------------------------------------------------------------------------

\listoftables % Print the list of tables

\cleartoverso % Force a break to an even page

%----------------------------------------------------------------------------------------
%	ACRONYMS
%----------------------------------------------------------------------------------------

\chapter{List of Acronyms}
\begin{acronym}\addtolength{\itemsep}{-\baselineskip}
  \acro{AIC}{Akaike Information Criterion}
  \acro{CAD}{Coronary Artery Disease}
  \acro{PRS}{Polygenic Risk Score}
  \acro{oPRS}{Optimal Polygenic Risk Score}
  \acro{CARDIOGRAMC4D}{Coronary ARtery DIsease Genome wide Replication and Meta analysis plus The Coronary Artery Disease Genetics}
  \acro{GWAS}{Genome wide Association Study}
  \acro{GLC}{Global Lipids Consortium}
  \acro{GIANT}{The Genetic Investigation of ANthropometric Traits}
  \acro{BMI}{Body Mass Index}
  \acro{MI}{Myocardial Infarction}
  \acro{kb}{kilobase}
  \acro{DNA}{Deoxyribonucleic acid}
  \acro{A}{Adenine}
  \acro{C}{Cytosine}
  \acro{T}{Thymine}
  \acro{G}{Guanine}
  \acro{locus}{specific genetic location}
  \acro{CNV}{copy number variant}
  \acro{InDel}{insertion/deletion}
  \acro{RNA}{ribonucleic acid}
  \acro{SNP}{single nucleotide polymorphism}
  \acro{LD}{linkage disequilibrium}
  \acro{FWER}{family wise error rate}
  \acro{FDR}{false disovery rate}
  \acro{PRDS}{positive regression dependence on subsets}
  \acro{OR}{odds ratio}
\end{acronym} % Include a List of Acronyms section using acronyms.tex where they are defined

\cleartoverso % Force a break to an even page

%----------------------------------------------------------------------------------------
%	COLOPHON
%----------------------------------------------------------------------------------------

\thispagestyle{empty} % Remove all headers and footers from this page

\vspace*{2em}
\renewcommand{\abstractname}{A note on notation}
\begin{abstract}

Throughout this thesis, the following conventions for notation are used.

\begin{enumerate}
	\item A hat ($\hat{.}$) denotes the estimator of a variable (i.e. $\hat{\beta}$ is the estimator of $\beta$).
	\item Underlining a variable ($\underline{.}$) implies that it is a $n$-vector, or $n \times 1$ dimensional matrix. (i.e. $\underline{Y} = \begin{bmatrix} 1 \\ 0 \\ 1 \end{bmatrix}$ is a $n=3$-vector).
	\item Bolding indicates a matrrix (i.e. $\mathbf{G} = \begin{bmatrix} 1 & \dots & 1 \\ \vdots & \ddots & \vdots \\ 1 & \dots & 1 \end{bmatrix}$)
\end{enumerate}


\end{abstract}
\vfill

%----------------------------------------------------------------------------------------
%	CONTENT CHAPTERS
%----------------------------------------------------------------------------------------

\mainmatter % Begin numeric (1,2,3...) page numbering

\chapterstyle{thesis} % Change the style of the Chapter header to that defined in structure.tex

\pagestyle{Ruled} % Include the chapter/section in the header along with a horizontal rule underneath

\chapter{Introduction}
\label{introduction}

As the efficiency and accuracy of rapid genome sequencing skyrockets, the potential for personalized therapies has made its way from science fiction to scientific reality. Using genetics to understand, diagnose, and eventually to predict illness is not a new idea; in recent years, however, technological ability and scientific understanding have advanced to such a point that researchers may predict risk for several diseases with reasonable confidence. Increasingly, variants in the human genome are being identified as being robustly linked to risk for complex illnesses such as heart disease [\small{cite 9p21}], obesity [\small{cite fto}], and schizophrenia [\small{cite something}]. However, much work remains to be done in order to create tools which may accurately predict individual disease risk from known and unknown genetic risk factors. In this thesis, we propose a novel extension to a well known methodology in order to better characterize disease risk from comorbid conditions using only summary statistics. 

In brief, we present preliminary evidence for the use of \ac{PRS}s in predicting \ac{CAD}. We use recently published summary statistics from a \ac{GWAS} conducted by the \ac{CARDIOGRAMC4D} consortium alongside evidence gathered by the \ac{GLC} and the \ac{GIANT} consortium for lipids and \ac{BMI} \textit{per say}. We use this data alongside previously identified variants to construct first a simplistic \ac{PRS} using only genome wide statistically signficant ($P_{Bonferonni} < 0.05$ or $q_{FDR}$ < 0.05) variants, then expand our search to variants which may not be as robustly linked to phenotype. [Cite storey, BH, and dudbridge]. We use an empirical maximation approach and several strategies of mathematical optimization in order to construct an \ac{oPRS}, then devise a novel technique for integrating information from co-morbid \ac{oPRS} diseases in order to better predict \ac{CAD} in four cohorts comprising approximately $n=12,000$ individuals

\section{Genetics of Coronary Artery Disease}

\ac{CAD} occurs when the major blood vessels supplying the heart become diseased or damaged, often leading to severe complications such as \ac{MI} and death. [cite review articles] \ac{CAD} is known to be a complex genetic disease with heritability estimated by twin studies between 40 and 60\%. [mcPherson 2016, twin studies paper] Several important variants have been indentified which have been shown to robustly increase risk to \ac{CAD} by altering lipid transporting pathways [cite LDLR], structural collegan bodies [TRIB 1??], and others factors. 


\begin{figure}[h]
\caption{Progression of the formation of plaque causing \ac{CAD}. Adapted from Gretch 2003.}
\centering
\includegraphics[width=0.8\textwidth]{Figures/cad.png}
\end{figure}

With heart disease and stroke the leading cause of perscription drug use in Canada as well as one of the leading causes of death and hospitalization [cite herat and stroke], the need to better understand, diagnose, and prevent this deadly disease is apparent.  In order to better understand the need for improved statistical methodologies, it is important to understand the large body of previous attempts to characterize the genetic determinants of \ac{CAD}

Despite some promising beginings, initial attempts to understand and explain \ac{CAD} through genetics were largely unsuccessful.[CITE] The first variant to be succesfully and robustly linked to risk for \ac{CAD} was the 9p21.3 locus. Discovered by a team of researchers at the Univeristy of Ottawa Heart institute, the allele consists of a 58 \ac{kb} region on chromosome 9 which was shown to be associated with \ac{CAD} in a population of 23,000 caucasion individuals. (\cite{McPherson2016})

\begin{figure}[h]
\caption{Fine mapping of the genomic interval on chromosome 9 associated with Coronary Heart Disease. Adapted from Mcpherson et al 2006}
\centering
\includegraphics[width=0.8\textwidth]{Figures/9p21.png}
\end{figure}

This initial success began the era of the \ac{GWAS}, explained in more detail in section \ref{gwas}.  Researchers across the globe began frantically searching for more loci with the hope of understanding and predicting complex disease; in that goal, the \ac{GWAS} has failed. (\cite{Visscher2012}) A number of important genetic markers for \ac{CAD} have been discovered, but often in small familial cases or with very low effect sizes. [Cite] As the dust settles and the low hanging fruit have been picked, common variants have been shown to explain approximately 28\% of the heritablity of \ac{CAD} [cite majid], yet a large portion remains to be accounted for. This has become known as the problem of ``missing heritabillity'' of complex disease; common genetic variants explain a relatively small portion of the total estimated heritability of a disease, therefore researchers must resort to ever more obscure and complex methods to attempt to explain the complex interactions between genetic elements in the human genome. [cite review paper] From pathway analysis to partitioned heritability to all kinds of arcane statistical procedures, researchers from across the globe have tried their hardest to shrink this gap between our knowledge and accurate prediction and understanding of complex disease. To this end, we develop our own methodology incorporating multiple sources of information for the more accurate prediction of clinical end points. 

\section{Genome Wide Association Studies} \label{gwas}

In order to properly introduce the model, however, the basic underpinnings must be explored and explained. Genome wide association studies seek to indetify associations between individual genotypes and disease phenotypes in a hypothesis free manner. In this section, the statistical model required to understand \ac{GWAS} is presented and explored.

\subsection{Primer on Genetics}

\ac{DNA} is a double helical molecule which encodes the genetic blueprints for the construction of proteins and other materials that make up every known living organism. \ac{DNA} is composed of three parts: a negatively charged phosphate group, a five carbon sugar \textit{deoxyribose}, and (usually) one of four nitrogen bases. It is these bases, \ac{A}, \ac{C}, \ac{T}, and \ac{G} and their combinations which are under investigation in a \ac{GWAS}. The specific combinations of these four bases in a \ac{locus} determine the product produced by the \ac{DNA}, and even a small change in this order can have large ramifications on the overall health, survival, and proper function of the organism. 

\subsection{Sequencing}

DNA sequencing is the process of ascertaining a particular individual's genotype by means of chemical identification of the bases present at predefined sites. [cite] These sites, whether they be a change in a single base called a \ac{SNP}, a variation in the number of tandem repeats of a small sequence named a \ac{CNV} or an \ac{InDel} of a sequence, may alter amino acid sequence, affect regulatory regions, or impact regulatory \ac{RNA} sequences.

\begin{definition}[Allele]
A specific form or subtype of a genetic locus. This could be one or more individual variations or a combination therof. 
\end{definition}

\begin{rem}
Allele frequency is the frequency at which a particular allele occurs in the population. I.e. for locus $A$ having $n$ different alleles, the true population allele frequency of allele $freq A_m \equiv \frac{A_m}{ \sum^n_{i=1} A_i}$, which is estimated in a sample population with a biased ratio estimator $freq \hat{A}_m \equiv \frac{\hat{A}_m}{ \sum^n_{i=1} \hat{A}_i}$  
\end{rem}

\subsection{Statistical Definition}

Consider a simple case control population where 1 defines case and 0 defines control. Define $\underline{\mathbf{Y}}$ as an $n$-vector where $n$ denotes the number of individuals in a population and $\underline{\mathbf{Y}}_i$ gives the individual's diesease staet. Additionally define $G$ as an $m \times n$ matrix where $m$ is the number of informative genotypic sites available with $\mathbf{G}_{ij}$ being the ``state'' (allele number) present at site $j, 1 \leq i \leq m, i \in \mathbb{Z}^+$ in individual $i, 1 \leq i \leq n, i \in \mathbb{Z}^+$.

$$ \begin{aligned} &\mathbf{\underline{Y}} &= \begin{bmatrix} Y_1 \\ \vdots \\ Y_n \end{bmatrix} \, \, \, \, \, \, \, \,\, \, \, \,\, \, \, \, \, \, \, \, \, \, \, \,\, \, \, \,\, \, \, \, &  \mathbf{G} &= \begin{bmatrix} G_{1,1} & \dots & G_{1, n} \\ \vdots & \ddots & \vdots \\ G_{m, 1} & \dots & G_{m, n} \end{bmatrix} \end{aligned} $$

In an additive genetic model, we define the phenotype $\underline{\mathbf{Y}}$ as a linear combination of $\mathbf{G}$ weighted by a vector of $\underline{\mathbf{\beta}}$ coefficicent vectors estimated by regression analysis and $\underline{\mathbf{\epsilon}}$ vector of errors. Express $\underline{\mathbf{Y}}$ such that

$$ \underline{\mathbf{Y}} = \underline{\mathbf{\beta}}' \mathbf{G} + \underline{\mathbf{\epsilon}} = \left( \sum^m_{i=1} \beta_i \mathbf{G}_{i, n} + \epsilon_n \right)' $$

$\underline{\mathbf{\beta}}$ and $\underline{\mathbf{\epsilon}}$ are approximated optimally by $\underline{\hat{\beta}}$ and $\underline{\hat{E}}$ in practice.

The purpose of a \ac{GWAS} is not only to estimate these genetic effects $\underline{\mathbf{\beta}}$ by $\underline{\hat{\beta}}$ but also to estimate their significance of association with phenotype vector $\underline{\mathbf{Y}}$ through a $\chi^2$ test and corresponding test statistic $m$-vector $\hat{\chi}^2$. The degrees of freedom of this test statistic will vary between methods and models, and so will be left as futher reading. 

By approximating $\underline{\chi^2}$ with $\hat{\underline{\chi^2}}$ and computing the corresponding $P$ values, reserchers are able to identify and quantify the effects of variants significantly ($P < 0.05$) associated with the phenotype. These results can be summarized in a Manhattan plot, named after the city of Manhattan with it's high rise buildings towering over the scenery. The $x$ axis of this plot is the genomic location (usually coloured by chromosome number) while the $y$ axis is the $\log_{10}$ of the $P$ value of association derived from $\hat{\underline{\chi^2}}$. 


\begin{figure}[H]
\caption{Example of a Manhattan plot from a \ac{GWAS} for \ac{CAD} performed by Shunkert et al. 2011}
\centering
\includegraphics[width=0.5\textwidth]{Figures/man_ex.jpg}
\end{figure}

\subsection{Multiple Comparisson Problem}

In such a set up, where $m$ may be in the millions and the threshold of significance is set to $P = \alpha = 0.05$, we encounter a canonical issue in statistical inference. Recall that $P$ is the probability of observing a $\chi^2$ statistic as large or larger than a specific $\chi^2_m$ assuming $H_0$ of no association is correct and $\alpha$ is the threshold at which a significant effect is declared. 

For the sake of description, we define $M$ as the number of \textit{independant} variants (that is, the effective number of variants which are not in \ac{LD})

\section{Polygenic Prediction of Complex Disease}

Refering to the defintions proposed in the previous section and recalling that 

\section{Polygenic Sliding Window Optimization}
\section{Summary} % Include the introduction chapter
\chapter{Methods}
\label{methods}


\section{Study Population}

There are four major cohorts used as a ``test''	 set in this study, comprising a total $n = 13371$.

\textbf{Ottawa Heart Genomics Study (OHGS):} Details of this cohort have been previously described \citep{Davies2012}. Both cases (1) and controls (0) were recruited from the Lipid Clinic at the University of Ottawa Heart Institute (UOHI). Cases with diabetes melliitus were entirely excluded. Cases were required to have at
least one of: a stenosis in a major epicardial vessel of at least 50\%; have had a percuteneous intervention (PCI); have had coronary artery bypass surgery (CABG); or have had a myocardial
infarction (MI). Earlier studies using this cohort examined the effect of age, and cases were required to be $\leq$ 55 years old for men and $\leq$ 65 years old for women. The controls were either healthy elderly patients recruited from the catherization laboratory or the UOHI; they had no stenosis $\geq 50\%$ in any major epicardial vessel and were required to be at minimum 65 years old for men and 70 years old for women. The study protocol was approved by the Human Research Ethics Board of the University of Ottawa Heart Institute and all participants provided informed consent.

\textbf{Cleveland Clinic (CCGB):} Cases and controls from the Cleveland Clinic Cohort followed the same collection procedure as outlined for OHGS except were collected at the catherization laboratory of the Cleveland Clinic.  


\textbf{Duke Cathgen Study (DUKE):} Both cases and controls were recruited from the catherization laboratory at Duke University. Cases were required to have at least one epicardial coronary vessel with $\geq$ 50\% stenosis while being at most 55 years old for males and 65 years old for females. Controls were asymptomatic and required to have $\leq 30$ \% stenosis in all coronary vessels. Subjects with diabetes melliitus, severe pulmonary hypertension or congenital heart disease were excluded. The study protocol was approved by the ethics committee and all participants provided informed consent.

\textbf{INTERHEART Cohort (ITH):} INTERHEART is a standardized case-control study of acute myocardial infarction from across the world. Only Caucasian participants were analyzed in this study due to issues with differing gene frequencies among ethnicities. Cases -- those showing acute MI, were age matched to within 5 years of controls who were community based individuals with no previous history or diagnosis of heart disease and exertional chest pain. The study protocol was approved by the ethics committees in all participating centers and all participants provided informed consent. A full list of ITH investigators is found at http://www.phri.ca/interheart/index2.html.

\section{Genotyping and Imputation}

SNP genotyping of the above cohorts was performed on either Affymetrix 6.0 or 500K chip arrays at the University of Ottawa Heart Institute using the recommended procedure from the manufacturer. They were processed as in \cite{Dandona2010,Schunkert2011}. Imputation was performed using IMPUTE2 and the August 2009 1000 Genomes reference panel. \citep{10.1371/journal.pgen.1000529}. Approximately 5.5 million \ac{SNP} passed quality control measures including info $> 0.5$, Hardy Weinburg Equilibrium $> 1 \times 10^{-6}$ and missingness $< 10\%$. 


\section{Training Populations}
\label{training}

This study additionally comprised two ``training'' populations which were used to estimate the $\hat{\beta}$ effects necessary for the construction of \ac{PRS}.

\textbf{GIANT Consortium: } \ac{GIANT} consortium attempts to identify genetic loci which may modulate human body size, height, and obesity. We use for this study their data on BMI predicting \ac{SNP} calculated from approximately $n = 123,865$ on close to 2M \acp{SNP}. Collection methodologies and specific information is outlined in \cite{Speliotes2010}.

\textbf{Global Lipids Consortium:} The Global Lipids Consortium estimates genetic effects in $n = 188,577$ individuals using whole genome and custom genotyping arrays. We use their estimated additive genetic effects for SNPs predicting \ac{TG}, \ac{HDLc}, and \ac{LDLc}. Collection methodologies and further information are outlined in \citep{Consortium2013}.

\textbf{CARDIoGRAMplusC4D:} \unsure{add in details for cardiogram, and make sure to pull down the missing part}


\section{Polygenic Prediction of CAD}

In the following analysis we primarily compare three different methods for constructing \ac{PRS} $\hat{S}$.

\subsection{Traditional Risk Score}

The first, which we denote as the ``traditional risk score'', or $\hat{S}_{TRS}$. This score uses the ``traditional'' approach of only using the highest confidence genome wide significant loci for \ac{CAD} in the construction of the score. We derive the estimated $\underline{\hat{\beta}}$ effects from \citep{TheCARDIoGRAMplusC4DConsortium2015}, whose methodology is described above. We only use the 212 variants from this section which have been shown to be FDR signficicant with $q < 0.05$ across the whole genome, as is common practice. Recall from the derivation leading up to equation \ref{score} that \ac{PRS} $S$ for individual $n$ is described as  $$ S = \sum^m_{i=1} \beta_i G_{ni} $$ Therefore for this score, we define $\hat{\beta}$ as a vector of length 212 with each of the estimated additive genetic effects derived from CARDIoGRAM plus C4D, and construct estimated score $\hat{S}$ for individual $n$ as $$ \hat{S}_{n, TRS} \equiv \sum^{212}_{i = 1} \hat{\beta}_i \mathbf{G}_{n, i} $$ This forms the basis for our first model.

\subsection{Cardiometabolic Risk Score}

The second score which we estimate is a novel derivation. We aim to use genetic information from several co-morbid conditions together to better explain \ac{CAD}. The motivation is that important signals may be spuriously insignificant in large meta analyses, or simply have too low effect to be accurately categorized as significant; taking information from co-morbid conditions allows researchers a wider span of information to integrate. 

We use meta data from four co-morbid conditions to estimate the genetic effects of these traits and \textbf{re-prioritize} variants with the intention of creating a minimal score for \ac{CAD} which better predicts the phenotype. 

First, we introduce some new notation. We denote $\underline{\hat{\beta}}_{LDLc}, \underline{\hat{\beta}}_{HDLc}, \underline{\hat{\beta}}_{TG}, \underline{\hat{\beta}}_{BMI}$ as the vectors of estimated effects for \ac{LDLc}, \ac{HDLc}, \ac{TG}, and \ac{BMI} respectively derived from the training sets outlined in section \ref{training}.

We separately order variants by their P value and say $m^*$ is the number of genome-significant significant ($q \leq 0.05$) hits found in each study. We take $1 \dots m^*$ from each data set and call this set of variants $\underline{G}^*$ for important genetic effects. We define the set $G^*$ as containing all genetic elements $\mathbf{G}_i$ such that $i$ is a part of our selected significant ordered set $1 \dots m^* $.

$$ \begin{aligned} G^* &\equiv \{ \mathbf{G}_i | i \in 1 \dots m^* \} &&&&&& m^* \, |  \, q < 0.05 \end{aligned}$$


We then take all genetic effects $i \in G^*$ and calculate a score based on these variants instead of the 212. We define this new \ac{CMB} score for any individual $n$ as $\hat{S}_{CMB}$:

$$ \hat{S}_{CMB} \equiv \sum_{i \in G^*_{LDLc}} \hat{\beta}_i \mathbf{G}_{n, i} + \sum_{i \in G^*_{HDLc}} \hat{\beta}_i \mathbf{G}_{n, i} + \sum_{i \in G^*_{TG}} \hat{\beta}_i \mathbf{G}_{n, i} + \sum_{i \in G^*_{TG}} \hat{\beta}_i \mathbf{G}_{n, i} $$

We use this new score to predict \ac{CAD}, with the hypothesis that incorporating several co-morbid conditions will better prioritize variants in order to achieve increased predictive accuracy. 
 
\unsure[inline]{this isn't exactly correct, make the list beforehand}

\improvement[inline]{talk about what to do in duplication cases in further work}

\subsection{Optimal Cardiometabolic Risk Score}

We further extend this $\hat{S}_{CMB}$ using \ac{oPRS} as introduced in section \ref{oPRS}. Instead of selecting $m^*$ to be all variants such that $q < 0.05$, we select all P values such that $P < T_{o}$ where $T_{o}$ is the optimal threshold found by iterating through $P$ value thresholds for score inclusion.

$$ \begin{aligned} G^* &\equiv \{ \mathbf{G}_i | i \in 1 \dots m^* \} &&&&&& m^* \, |  \, P < T_{o} \end{aligned}$$

And similarly construct our optimal cardiometabolic risk score as before:


$$ \hat{S}_{oCMB} \equiv \sum_{i \in G^*_{LDLc}} \hat{\beta}_i \mathbf{G}_{n, i} + \sum_{i \in G^*_{HDLc}} \hat{\beta}_i \mathbf{G}_{n, i} + \sum_{i \in G^*_{TG}} \hat{\beta}_i \mathbf{G}_{n, i} + \sum_{i \in G^*_{TG}} \hat{\beta}_i \mathbf{G}_{n, i} $$

This forms the new optimal score for testing.

\section{Statistical Analysis}

\improvement[inline]{add in as go}

\section{Computational Resources}

All analyses were performed at the Center for Advanced Computing, a large scale Red Hat Enterprise linux parallel computing cluster used to quickly analyze large sets of data. All analysis was parallelized either using inbuilt libraries or OpenMPI standards. 

Analyses were performed in R version 3.2.3 (https://www.r-project.org/), Python legacy version 2.7.9 (https://www.python.org/), Plink (http://pngu.mgh.harvard.edu/~purcell/plink/), and GCTA (http://cnsgenomics.com/software/gcta/).

All analysis was logged and stored securely and anonymously; back ups of all data were made and encrypted.

All analysis was version controlled using git + github and this thesis is entirely reproducible. Any code available upon request. 
 % Include the first content chapter
\chapter{Results}
\let\cleardoublepage\clearpage 

General characteristics of the study population are displayed in Table \ref{pop}.
\tabularnewline
\tabularnewline

\begin{table}[bp]
\centering
\begin{tabular}{llllll}
\hline
                   & All Participants &  & Cases       &  & Controls    \\ \hline
n                  & 9663             &  & 5831        &  & 3832        \\
Age$^1$ (years)       & 62.8 $\pm$ 12.3      &  & 56.2 $\pm$ 10.1 &  & 73.0 $\pm$ 7.4  \\
Smoke Current (\%) & 29.6             &  & 36          &  & 20          \\
Male (\%)          & 65.3             &  & 76.7        &  & 47.9        \\
Obese $^2$ (\%)       & 29               &  & 35.1        &  & 19.7        \\
BMI (kg/$m^2$)        & 28.1 $\pm$ 5.3       &  & 28.9 $\pm$ 5.3  &  & 26.7 $\pm$ 4.9  \\
TG $^3$ (mmol/L)      & 1.46 $\pm$ 1.47      &  & 1.66 $\pm$ 1.70 &  & 1.18 $\pm$ 0.99 \\
HDLc$^3$ (mmol/L)     & 1.27 $\pm$ 0.44      &  & 1.13 $\pm$ 0.39 &  & 1.46 $\pm$ 0.44 \\
LDLc$^3$ (mmol/L)     & 3.29 $\pm$ 1.08      &  & 3.18 $\pm$ 1.17 &  & 3.43 $\pm$ 0.93 \\ \hline
\end{tabular}
\caption[General Population descriptions.]{General Population description. All values are expressed as mean $\pm$ one standard deviation unless otherwise noted. $^1$  Age represents age at consent for controls and age at diagnosis for cases
 $^2$ Obesity is defined as having a BMI of greater or equal to 30 kg/m2 at time of collection $^3$T G (triglyceride), LDLc (low density lipoprotein cholesterol), HDLc (high density lipoprotein cholesterol).}
\label{pop}
\end{table}


\section{Traditional Risk Score}

The traditional risk score was significantly ($P < 2.2 \times 10^{16}$) associated with case/control status in all cohorts when adjusted for principal components to control for population stratification. \citep{Price2006,Zhang2013}. On average, scores between cases and control differed by $5.06 \times 10^{-4} \pm 1.73 \times 10^{-4}$. Full distributions of the score dependent on case control status are presented in Supplementary Figure 4.1. $\hat{S}_{TRS}$ predicted CAD status better than chance in all cohorts. Area under the receiver operator characteristic curve was calculated for each model and meta-analysed. The resulting AUC $\pm$ 95\% confidence intervals are displayed in Figure \ref{trs_meta}.

\begin{figure}[h]
\label{trs_meta}
\centering
\includegraphics[width=0.8\textwidth]{Figures/trs_meta.png}
\caption[Random effects meta analysis of AUC ROC from six cohorts.]{\textbf{Random effects meta analysis of AUC ROC from six cohorts.} Logistic regression models were constructed with $\hat{S}_{TRS}$ and the first two principal components adjusting for population stratification were used to predict \ac{CAD}. Various thresholds for false positives and true positives were used to construct \ac{ROC} curves. Area under these curves were estimated along with error. The associated effects were analyzed assuming cohorts were random effects and an overall effect was derived.}
\end{figure}

Logistic regression was used to predict observations following the model. Summary statistics from the models employed are displayed in Table \ref{trs}

$$ CAD = X_0 + \hat{\beta}_1 \hat{S}_{TRS} + \hat{\beta}_2 PC_1 + \hat{\beta}_3 PC_3 + \epsilon $$

\begin{table}[H]
\centering

\begin{tabular}{llllll}
\hline
Cohort           & OR        & SE       & $R^2$      & AIC      & AUC       \\ \hline
Cleveland Clinic & 153.89908 & 36.42776 & 0.01621871 & 1877.724 & 0.5740551 \\
Duke University  & 255.78537 & 33.44304 & 0.04713465 & 2335.567 & 0.6129263 \\
Interheart       & 83.95637  & 46.45943 & 0.01695232 & 1172.994 & 0.5554871 \\
OHGS A2          & 310.49752 & 34.52749 & 0.07650691 & 2558.277 & 0.6391285 \\
OHGS B2          & 259.59352 & 24.94511 & 0.07425403 & 3681.768 & 0.6339916 \\
OHGS C2          & 276.96733 & 45.92012 & 0.05491302 & 1318.919 & 0.6193695 \\ \hline
\end{tabular}
\caption[Summary statistics from Logistic association model for $\hat{S}_{TRS}$.]{\textbf{Summary statistics from Logistic association model.} $\hat{S}_{TRS}$ along with the first two principal components to adjust for population stratification were used to predict \ac{CAD}. OR corresponds to the odds ratio of $\hat{S}_{TRS}$ along with its standard error (SE). $R^2$ corresponds to NagelKerke's Pseudo-$R^2$, while AIC corresponds to Akaike Information Criterion, a measure of model fit. AUC corresponds to the area under the \ac{ROC} curve as derived in the pROC package in R.}
\label{trs}
\end{table}

An \ac{AUC} $\geq 0.5$ indicates that the model predicts CAD better than chance. The overall random effects meta analyzed \ac{AUC} was 0.61 $\pm 0.03$ for $\hat{S}_{TRS}$, with an average NagelKerke's Pseudo-$R^2$ of $0.047$. Interheart was the worst fit model, with NagelKerke's Pseudo-$R^2 = 0.017$ and \ac{AUC} $= 0.555$, barely predicting above chance. This is in contrast to the OHGS cohorts, which consistently fit better than Cleveland Clinic, Duke, or Interheart. As increasing the number of \acs{SNP} in the score will always increase the fit of the model, we compare the 202 FDR significant loci to randomly selected loci in 1000 bootstraps in Figure \ref{b2_perm}. 

The score remains significantly associated when adjusted for individual's sex, a known cardiovascular risk factor. The predictive accuracy of the model significantly increases after inclusion of sex, as would be expected. The random effects meta analysis \ac{AUC} for all six cohorts becomes $0.69[0.62, 0.76]$. 

In \ac{OHGS} B2, sex significantly ($P = 0.00226$) interacts with the effect of the risk score. This shows that sex modulates the effect of genetics in this cohort. The remainder of the cohorts do not interact.


\begin{figure}[H]
\centering
\includegraphics[width=0.5\textwidth]{Figures/b2.png}
\label{b2_perm}
\caption[$\hat{S}_{TRS}$ predicts \ac{CAD} significantly better than 1000 \ac{PRS} constructed with an equal number of \acp{SNP}.]{\textbf{$\hat{S}_{TRS}$ predicts \ac{CAD} significantly better than 1000 \ac{PRS} constructed with an equal number of \acp{SNP}.} 202 \acp{SNP} were randomly selected from post quality control imputed \acp{SNP} and a \ac{PRS} was constructed using summary information from \cite{TheCARDIoGRAMplusC4DConsortium2015} and Nagelkerke's Pseudo $R^2$ was plotted against frequency. The red line denote the Nagelkerke's $R^2$ of the true \ac{TRS} which predicts significantly $P \approx 0$ better than the random model.}
\end{figure}

These results are in line with literature values. Those in the top quintile of the \ac{PRS} were 70\% more likely to have \ac{CAD} than not (2045 cases vs 1198 controls). Similarly, those in the bottom quintile were  5\% less likely to be diagnosed with \ac{CAD} than not (1576 cases vs 1659 controls).



\section{Cardiometabolic Risk Score}

We add in each co-morbid score in a stepwise manner. $\hat{S}_{CMB; 1}$ uses just the information available for CAD and is equivalent to the above section.  $\hat{S}_{CMB; 2}$ uses information from CAD and BMI, while $\hat{S}_{CMB; 3}$ uses CAD, BMI, and LDLc. $\hat{S}_{CMB; 4}$ uses information from CAD, BMI, LDLc, and TG, while $\hat{S}_{CMB; 5}$ uses CAD, BMI, LDLc, TG, and HDLc. Note that in this section we restrict our analysis to the \ac{OHGS} cohorts along with Cleveland Clinic due to the lack of quality in lipid data present in the other cohorts. 

The summary statistics of this analysis are presented in Table \ref{cmd-sum}. Higher genetic risk scores were uniformly and significantly ($P < 2.2 \times 10^{-16}$) associated with increased risk of \ac{CAD}.

\begin{table}[H]
\centering
\begin{tabular}{lllllll}
\hline
Score & Cohort           & OR        & SE       & R2         & AIC      & AUC       \\ \hline
$\hat{S}_{CMB; 1}$     & OHGS A2          & 310.49752 & 34.52749 & 0.07650691 & 2558.277 & 0.6391285 \\
      & OHGS B2          & 259.59352 & 24.94511 & 0.07425403 & 3681.768 & 0.6339916 \\
      & OHGS C2          & 276.96733 & 45.92012 & 0.05491302 & 1318.919 & 0.6193695 \\
      & Cleveland Clinic & 153.89908 & 36.42776 & 0.01621871 & 1877.724 & 0.5740551 \\ \hline
$\hat{S}_{CMB; 2}$     & OHGS\_A2         & 252.9639  & 25.29332 & 0.09108544 & 2547.617 & 0.6555183 \\
      & OHGS\_B2         & 267.348   & 22.48745 & 0.09187763 & 3654.756 & 0.6501946 \\
      & OHGS\_C2         & 280.9237  & 38.98146 & 0.07687197 & 1300.585 & 0.6455279 \\
      & Cleveland        & 337.3714  & 35.18263 & 0.08387982 & 1791.596 & 0.6659644 \\ \hline
$\hat{S}_{CMB; 3}$     & OHGS\_A2         & 236.0145  & 26.17239 & 0.07656732 & 2570.063 & 0.6429851 \\
      & OHGS\_B2         & 255.7401  & 24.09888 & 0.0767165  & 3688.513 & 0.6371506 \\
      & OHGS\_C2         & 284.9796  & 41.16138 & 0.07121422 & 1305.337 & 0.640201  \\
      & Cleveland        & 344.9182  & 37.87978 & 0.07572764 & 1802.024 & 0.6546671 \\ \hline
$\hat{S}_{CMB; 4}$     & OHGS\_A2         & 266.5048  & 29.24856 & 0.07786803 & 2568.062 & 0.6442205 \\
      & OHGS\_B2         & 270.8893  & 26.45495 & 0.07264458 & 3697.51  & 0.6330786 \\
      & OHGS\_C2         & 298.9828  & 44.83843 & 0.06638254 & 1309.38  & 0.6337181 \\
      & Cleveland        & 372.189   & 41.80231 & 0.07218274 & 1806.542 & 0.6509422 \\ \hline
$\hat{S}_{CMB; 5}$     & OHGS\_A2         & 297.0165  & 34.10419 & 0.07234597 & 2576.54  & 0.6401786 \\
      & OHGS\_B2         & 299.0699  & 30.76742 & 0.06720315 & 3709.488 & 0.6280858 \\
      & OHGS\_C2         & 323.9172  & 50.90267 & 0.06088933 & 1313.958 & 0.6275034 \\
      & Cleveland        & 410.9963  & 48.57387 & 0.06493754 & 1815.742 & 0.6436849 \\ \hline
\end{tabular}
\label{cmd-sum}
\caption[Summary statistics from Logistic association model for $\hat{S}_{CMD}$.]{\textbf{Summary statistics from Logistic association model.} $\hat{S}_{CMD; x}$ along with the first two principal components to adjust for population stratification were used to predict \ac{CAD}. OR corresponds to the odds ratio of $\hat{S}_{TRS}$ along with its standard error (SE). $R^2$ corresponds to NagelKerke's Pseudo-$R^2$, while AIC corresponds to Akaike Information Criterion, a measure of model fit. AUC corresponds to the area under the \ac{ROC} curve as derived in the pROC package in R.}
\end{table}

In permutation analyses, each of the scores performed significantly better than an equivalent number of randomly selected \acp{SNP} in 1000 bootstraps. The random effects meta analysis \ac{AUC} values were $0.65 [ 0.64, 0.67], 0.64[0.63, 0.66], 0.64[0.63, 0.65], 0.63[0.62, 0.65]$ for scores 1 through 5 respectively. Interestingly, the more \acp{SNP} added in, the worse the model was at predicting the phenotype. The \ac{AIC} is also uniformly smaller in score 2 than in subsequent scores, meaning that this model fit the data the best. Persons in the upper quintile of this score were 81\% more likely (1725 cases vs 953 controls) to have \ac{CAD} than not (compared to 70 \% for the previous score) and people in the bottom quintile were 15.7\% less likely (1240 cases vs 1471 controls) to have \ac{CAD} than having it. There was also a substantive increase in NagelKerke's Pseudo $R^2$ in the second score compared to any other, especially in the Cleveland cohort. 

The score maintained its significance even after inclusion of biologically relevant covariates such as gender and smoking status. Additionally, the predictive accuracy of the score increased substantially after the inclusion of these covariates. After inclusion of individual's sex, a known cardiovascular risk factor, predictive accuracy increased substantially. The random effects meta analysis \ac{AUC} values were increased to $0.69[0.62, 0.76], 0.74[0.67, 0.81], 0.73 [0.66, 0.81], 0.73[0.66, 0.81], 0.73[0.66, 0.80]$ for $\hat{S}_{CMB; 1}$ through $\hat{S}_{CMB; 5}$ respectively. Again it appears as though $\hat{S}_{CMB; 2}$ is the best predictive model.


We additionally investigated whether sex significantly modulates the effect of  $\hat{S}_{CMB; .}$ and found no significant interactions. 

\subsection{Optimal Cardiometabolic Risk Score}

When optimal scores were derived, the thresholds described in Table \ref{oprs} were observed.

\begin{table}[H]
\centering

\begin{tabular}{lllll}
\hline
                  & \textbf{LDLc} & \textbf{HDLc} & \textbf{TG} & \textbf{BMI} \\ \hline
OHGS\_A2 & 0.0001        & 0.0055        & 0.1743      & 1            \\
OHGS\_B2 & 0.2484        & 0.0999        & 0.0002      & 1            \\
OHGS\_C2 & 0.1299        & 0.0085        & 0.1528      & 1            \\
CCGB\_2  & 0.1807        & 0.2039        & 0.004       & 1            \\ \hline
\end{tabular}

\caption[Optimal \ac{PRS} $P$-value thresholds.]{\textbf{Optimal \ac{PRS} $P$ value thresholds ($T_o$) derived for each trait in each cohort.} 2500 thresholds were created between $P = 0.0001$ and $P = 0.25$ and used as inclusion threshold $T$ for \ac{PRS}. These \acp{SNP} were used to construct a \ac{PRS}, which was used alongside the first two principal components and sex as covariates to predict \ac{CAD}. Their respective $P$-values of association were recorded and the maximal $-\log_{10} P$-value of association was used as the optimal threshold $T_0$.}
\label{oprs}
\end{table}


The optimal $P$ value cutoff threshold $T_o$ for \ac{BMI} was found to be $1$ for all cases as demonstrated in \ref{oprs_bmi}. We discuss this result further below, however, for now we exclude \ac{BMI} from the analysis of optimal risk scores, and opt to use just scores for the lipid traits.

\begin{figure}[H]
\centering
\includegraphics[width=0.5\textwidth]{Figures/PRSice_HIGH-RES_PLOT_2016-04-23.png}
\label{oprs_bmi}
\caption[Optimal $P$-value inclusion threshold for \ac{BMI}.]{\textbf{High definition assocaition plot for \ac{BMI} \acp{SNP} with \ac{CAD}.}2500 thresholds were created between $P = 0.0001$ and $P = 0.25$ and used as inclusion threshold $T$ for \ac{PRS}. These \acp{SNP} were used to construct a \ac{PRS}, which was used alongside the first two principal components and sex as covariates to predict \ac{CAD}. Their respective $P$-values of association were recorded and the maximal $-\log_{10} P$-value of association was used as the optimal threshold $T_0$. There was no maximal $-\log_{10} P$ value threshold less than 0.25, consistent with the model shown in figure \ref{pi0} with low $\pi_0$, or proportion of truly null \acp{SNP}.}
\end{figure}

Constructing a logistic regression model, we find that our cumulative score (combining optimal scores from all three traits) is significantly ($P < 2.2 \times 10^{-16}$.) associated with \ac{CAD} status, and those who have a higher \ac{PRS} tend to have higher risk for \ac{CAD}. We detail summary statistics for this association in table \ref{oprs_ss}. 



\begin{table}[H]
\centering

\begin{tabular}{lllllll}
\hline
Study & $n_{SNPs}$ & OR         & SE      & $R^2$     & AIC     & ROC    \\ \hline
OHGS\_A2 & 439971 & $2.2\times10^4$  & $1.3\times10^3$ & 0.2451 & 2313    & 0.7478 \\
OHGS\_B2 & 790128 & $3.42\times10^4$ & $1.5\times10^3$ & 0.3377 & 3061.9  & 0.7963 \\
OHGS\_C2 & 622050 & $3.34\times10^4$ & $2.5\times10^3$ & 0.3009 & 1095.1  & 0.7929 \\
CCGB\_2 & 847335 & $3.78\times10^4$ & $2.3\times10^3$ & 0.2836 & 1517.21 & 0.7953 \\ \hline
\end{tabular}
\label{oprs_ss}
\caption[Summary statistics from Logistic association model for $\hat{S}_{oCMD}$.]{\textbf{Summary statistics from Logistic association model.} $\hat{S}_{oCMD}$ along with the first two principal components to adjust for population stratification were used to predict \ac{CAD}. OR corresponds to the odds ratio of $\hat{S}_{TRS}$ along with its standard error (SE). $R^2$ corresponds to NagelKerke's Pseudo-$R^2$, while AIC corresponds to Akaike Information Criterion, a measure of model fit. AUC corresponds to the area under the \ac{ROC} curve as derived in the pROC package in R.}
\end{table}

The \ac{oPRS}, comprising optimal scores for \ac{LDLc}, \ac{HDLc}, and \ac{TG} predicts \ac{CAD} with a very high accuracy, explaining between 24.5 and 33.8 percent of variance in \ac{CAD}. The random effects meta analysis of \ac{AUC} of the \ac{ROC} curve reveals an overall \ac{AUC} of $0.78 [0.75, 0.81]$, an extremely high value (Figure \ref{oprs_meta}). 

\begin{figure}[H]
\label{oprs_meta}
\caption{Random effects meta analysis of \ac{oPRS} predicting risk for \ac{CAD}}
\centering
\includegraphics[width=0.5\textwidth]{Figures/oprs_meta.png}
\end{figure}


Though the \ac{AUC} varies significantly between the cohorts, it is significantly higher than any observed in either the \ac{TRS} or the \ac{CMB} scores. However, as noted in table \ref{oprs_ss}, each \ac{oPRS} comprises several hundred thousand \acp{SNP}, and it is difficult to assess whether the increase in predictive accuracy is simply a consequence of increasing the number of \acp{SNP} used to predict the phenotype, or whether it is a true biological occurrence. When an equal number of \acp{SNP} were selected at random in 1000 bootstraps, it was found that the $R^2$ of our \ac{oPRS} was not significantly $P > 0.05$ different than that obtained through permutation.


\subsection{Model Comparisons}

When the cardiometabolic $\hat{S}_{CMB; 2}$, determined to be the best predictive candidate score for \ac{CAD} was compared against the \ac{TRS} score in 1000 random bootstraps in all cohorts combined, it was found to have a significantly larger \ac{AUC} ($P < 2.2 \times 10^{-16}$). A similar result was found when the \ac{oPRS} was compared to both the \ac{TRS} and the \ac{CMB} scores. % Include the second content chapter
%\include{Chapters/chapter3} % Include the third content chapter

\backmatter

\chapterstyle{default} % Reset the chapter style back to the default used for non-content chapters

%----------------------------------------------------------------------------------------
%	BIBLIOGRAPHY
%----------------------------------------------------------------------------------------

\bibliographystyle{plainnat} % Use the plainnat bibliography style

\bibliography{bibliography} % Use the bibliography.bib file as the source of references

%----------------------------------------------------------------------------------------
%	INDEX
%----------------------------------------------------------------------------------------

\printindex % Print the index

%----------------------------------------------------------------------------------------

\end{document}