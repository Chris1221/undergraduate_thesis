\chapter{Conclusions}

In this thesis submitted to partially fulfill the requirements of an honours undergraduate degree in biomedical science, we introduce and validate a novel methodology for combining summary information from \ac{GWAS} in order to better predict complex disease from co-morbid conditions and phenotypes. We introduce \ac{GWAS} and their statistical properties, leading into a discussion of \ac{PRS}. We build upon these definitions to create our new model, which combines scores across several conditions. We find that our new \ac{CMB} score predicted \ac{CAD} significantly better than did the \ac{TRS}. We additionally constructed an optimal \ac{PRS} (\ac{oPRS}), which predicted between 60 and 80 \% of the variance in \ac{CAD} attributable to genetics, but was not significantly different from a random bootstrap. We have also shown that the characteristics of \ac{oPRS} $P$ value threshold selection in \ac{BMI} are consistent with those observed when a relatively high number of low effect \acp{SNP} affect the phenotype, lending evidence to this notion. 

Our novel scoring technique represents a significant improvement over the traditional polygeneic prediction of \ac{CAD}. However, much theoretical research is needed to validate and explore the trends observed. We believe that by pushing further into this area of knowledge, eventually \ac{PRS}, a widely used but often naive methodology, will be improved to the point of clinical utility. Using further iterations of methods like ours, eventually clinicians may be able to predict patient's clinical phenotypes with a high degree of accuracy from a small number of \acp{SNP}.